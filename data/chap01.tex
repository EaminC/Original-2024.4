% !TeX root = ../thuthesis-example.tex

\chapter{引言}


\section{研究背景}
 在现代信息检索系统设计中,待检索物品、用户数量及交需求规模急剧增大,使得准确识别用户偏好、快速精准地为用户提供信息检索结果成为需要格外着重优化的目标。早在公元前三世纪,人类便开始为产生的大量信息设计信息检索系统。例如人类第一座国际性图书馆----古希腊的亚历山大图书馆,涵盖了当时世界上绝大多数书籍或复制品,便已设计了初具雏形的信息分类检索系统\cite{1}。随着交互规模急剧及更新频率加快,这一被动式信息检索系统已经难以满足设计初衷,因此为减轻用户检索负担同时也继续满足高效精准的信息检索的需要,R.Armstrong\cite{2}等在1995年第一次提到了一种主动式信息检索系统,也是如今推荐系统的原始形态。随着移动互联网的发展以及用户-物品交互频次的进一步提升,业界急需更高效更精准的推荐系统技术。例如,2024年美团公司公布的财报\cite{3}中提到,其外卖业务全年产生了21.89亿次用户-物品交互,同比增长了23.9\%。信息数量的增长以及交互类别的快速更新也对推荐系统的设计带来了巨大且全新的挑战。
 
为更好地应对交互数据过载带来的推荐系统负荷、更精准地识别用户意图同时更高效且贴近用户真实交互行为的推荐结果,研究者们开始探究推荐系统(Recommender System),希望通过数据挖掘的方式从大量诸如用户评分、用户交互序列等行为数据得到用户偏好的特征表示,并对用户未来的交互行为进行预测。例如在电影网站里,通过收集既往用户对电影的评分对用户尚未观看的电影可能做出的评分进行预测从而给出推荐列表\cite{harper2015movielens},以帮助用户更快地找到偏好的电影。在城市生活数字化领域,用户与POI(Point Of Interest,即用户可能产生交互的兴趣点)的交互,例如居民在吃穿住行方面的大量交互行为,可以用于预测其未来行为,从而为城市生活带来便利。


以上种种应用都极大程度的依赖过往交互数据的质量及数量,力求推荐系统能够预测出与推真实交互行为接近的推荐结果。移动互联网时代的大量交互记录为基于行为统计的未来预测技术提供了数据基础。而得益于自然语言处理技术的蓬勃发展,基于内容语义理解的推荐技术在近年来逐渐受到重视。为此,本工作提出了一种利用大语言模型对内容及行为进行融合的混合推荐系统框架及实现。

 
\section{研究现状}
根据推荐系统数据类型及建模方式的不同,推荐系统大致上可以分为基于用户交互的推荐系统(Interaction-Based RS)、基于内容的推荐系统(Content-Based RS)以及混合推荐系统(Hybrid RS)\cite{ko2022survey}。随着例如ChatGPT等的一系列大语言模型的发布,依靠着其对自然语言的出色理解能力以及对先验知识出色的泛化能力,逐渐形成了一种基于大语言模型技术逐渐成型的基于智能体的推荐系统(Agent-Based RS)。其中基于用户交互的推荐系统的基础是大量的用户与物品交互记录以及用户对物品的评分,比较具有代表性的一类算法是基于用户交互评分的协同过滤算法(Collaborate Filtering, CF)\cite{resnick1994grouplens},能够识别出有相似交互行为的用户,并将相似用户之间差异化的剩余物品进行交叉推荐;或反之对有相似用户群体的物品产生交互的用户之间进行交叉推荐。这一技术及其依赖于既有交互数据的数量以及质量,因此在推荐系统搭建初期会面临及其严峻的冷启动问题,同时面对跨领域的推荐也会有先验数据缺乏的问题。协同过滤算法作为最具影响力的推荐系统算法之一也逐渐有了很多优化和变体,例如基于模型(Model-Based CF)\cite{carlkadie1998empirical}及基于记忆(Memory-Based CF)的协同过滤算法,后者又可以细分为基于用户(User-Based CF)\cite{goldberg1992using}或物品(Item-Based CF)\cite{sarwar2001item}的不同变体。与之相对应的是基于内容语义理解的推荐系统,这类推荐系统的核心是对推荐系统中的物品以及用户本身作为内容进行特征提取以及向量化,从内容本身获取相似性。这一类算法从原理上就一定程度上规避了基于用户交互所产生的交互数据依赖从而能够对于新的物品以及新的用户基于语义理解能够有一定的泛化能力。同时由于其极大程度上依赖特征提取的质量,因此特征提取质量不佳会极大程度影响最终推荐系统的推荐效果。
  
随着推荐系统技术的逐步发展,一些研究者\cite{ko2022survey}逐渐意识到了两种推荐系统算法各自有独到的优势以及劣势,因此一个更高质量的推荐系统应当能够很好的兼顾用户行为以及推荐系统数据库中的用户及物品的内容信息,因此可以对基于不同原理的推荐系统进行一些融合对齐。常见的方法例如对不同的推荐系统生成的评分进行加权(Weighted)、根据一定的规则进行模型的调用(Weighted)以及推荐系统之间的级联(Cascaded)或混合(Mixed)的方式对多个推荐系统进行模型级别或特征级别的整合。随着大语言模型技术的成熟,可以利用基于大语言模型的智能体的出色推理能力,在推荐系统中深入理解推荐内容的语义、深入理解用户画像同时理解用户交互的情景从而给出更灵活精准的推荐结果。 


\subsection{基于用户交互的推荐系统}

Paul Resnick\cite{resnick1994grouplens}等人最早将协同过滤这一概念引入至推荐系统技术之中。具体而言,协同过滤的设计初衷大致是出于以下对交互行为的假设:具有相似性的用户会对相似的物品感兴趣,而反之具有相似性的物品也会吸引相似的用户产生交互行为。这一朴素但直观的思想贯穿了协同过滤算法以及之后的几种变体。最常见的协同过滤算法主要有基于用户(User-Based CF)\cite{goldberg1992using}、基于物品(Item-Based CF)\cite{sarwar2001item}及基于模型(Model-Based CF)\cite{carlkadie1998empirical}的协同过滤算法。\\
\textbf{基于用户的协同过滤算法}\cite{goldberg1992using} \quad 在已有大量用户物品交互评分数据的情况下,基于用户的协同过滤首先得到各个用户对所有物品的交互评分记录作为向量表示,计算用户间的相似度,对有更高相似度之间的用户之间采取交叉推荐,其中相似度的计算参考了统计学中常用的皮尔森相似度\cite{pearson1896vii}。

具体而言用户\( U_a \) 和 用户\( U_b \) 的相似度可以表示为:

\begin{equation}
\text{sim}(U_a, U_b) = \frac{\sum_{i \in I_{a} \cap I_{b}} (S_{a,i} - \bar{S}_a)(S_{b,i} - \bar{S}_b)}{\sqrt{\sum_{i \in I_{a} \cap I_{b}} (S_{a,i} - \bar{S}_a)^2 \sum_{i \in I_{a} \cap I_{b}} (S_{b,i} - \bar{S}_b)^2}}\label{eq:appendix-equation}
\end{equation}
其中\( \text{sim}(U_a, U_b) \) 是用户 \( U_a \) 和用户 \( U_b \) 之间的相似度,\( S_{a,i} \) 和 \( S_{b,i} \) 分别是用户 \( U_a \) 和 \( U_b \) 对项目 \( i \) 的评分,\( \bar{S}_a \) 和 \( \bar{S}_b \) 是用户 \( U_a \) 和 \( U_b \) 的平均评分,\( I_a \cap I_b \) 是用户 \( U_a \) 和 \( U_b \) 都评分的项目集合。在定义了相似度之后对于单个用户就可以得到其邻域\( N \),可以基于领域中的用户对当前用户未有交互的物品评分进行预测,从而给出推荐列表。

具体而言,用户 \( U_a \) 对项目 \( i_j \) 的预测评分可以表示为:

\begin{equation}
\hat{S}_{a,j} = \bar{S}_a + \frac{\sum_{b \in N(a)} \text{sim}(U_a, U_b) \cdot (S_{b,j} - \bar{S}_b)}{\sum_{b \in N(a)} |\text{sim}(U_a, U_b)|}\label{eq:user-based-prediction}
\end{equation}
其中, \( \hat{S}_{a,j} \) 是用户 \( U_a \) 对项目 \( i_j \) 的预测评分\( N(a) \) 是与用户 \( U_a \) 最相似的一组用户集合。\\
\textbf{基于物品的协同过滤算法}\cite{sarwar2001item} \quad  除了基于用户的协同过滤算法以外,与之对应的还有基于物品的协同过滤算法。两种算法的核心思想是相似的,区别是基于物品的协同过滤将物品用用户交互的向量表示,而在实际的推荐系统环境中,用户数量是远远大于物品数量的,因此在减小向量数量的同时,增大了向量维度,具体使用哪一种依赖于具体的数据场景。在将物品向量化后可计算物品之间的相似度。

\begin{figure}[h!]
    \centering
    \includegraphics[width=1\linewidth]{figures/CF.pdf} % 调整scale参数以适应页面
    \caption{协同过滤示意图}
    \label{fig:enter-label}
\end{figure}
具体而言,物品 \( I_i \) 和物品 \( I_j \) 的相似度可以表示为:

\begin{equation}
\text{sim}(I_i, I_j) = \frac{\sum_{u \in U_{i} \cap U_{j}} (S_{u,i} - \bar{S}_i)(S_{u,j} - \bar{S}_j)}{\sqrt{\sum_{u \in U_{i} \cap U_{j}} (S_{u,i} - \bar{S}_i)^2 \sum_{u \in U_{i} \cap U_{j}} (S_{u,j} - \bar{S}_j)^2}}\label{eq:item-similarity}
\end{equation}
其中 \( \text{sim}(I_i, I_j) \) 是物品 \( I_i \) 和物品 \( I_j \) 之间的相似度,\( S_{u,i} \) 和 \( S_{u,j} \) 分别是用户 \( U_u \) 对物品 \( I_i \) 和物品 \( I_j \) 的评分,\( \bar{S}_i \) 和 \( \bar{S}_j \) 是物品 \( I_i \) 和物品 \( I_j \) 的平均评分,\( U_{i} \cap U_{j} \) 是对物品 \( I_i \) 和 \( I_j \) 都评分的用户集合。在定义了相似度之后,对于单个物品就可以得到其邻域 \( N \),可以基于邻域中的物品对当前用户未有交互的物品评分进行预测,从而生成推荐列表。

具体而言,用户 \( U_u \) 对物品 \( I_i \) 的预测评分可以表示为:

\begin{equation}
\hat{S}_{u,i} = \frac{\sum_{j \in N(i)} \text{sim}(I_i, I_j) \cdot S_{u,j}}{\sum_{j \in N(i)} |\text{sim}(I_i, I_j)|}\label{eq:item-based-prediction}
\end{equation}
其中, \( \hat{S}_{u,i} \) 是用户 \( U_u \) 对物品 \( I_i \) 的预测评分,\( N(i) \) 是与物品 \( I_i \) 最相似的一组物品集合。可以看到两种推荐算法的核心思想是对称的,区别在于具体数据应用场景中物品和用户哪一个维度更高,更有利于快速计算。\\
\textbf{基于模型的协同过滤算法}\cite{carlkadie1998empirical} \quad 在实际的推荐系统中,随着用户数量以及物品数量急剧增大,实际的交互矩阵是极为稀疏的,使得基于物品以及基于用户的协同过滤算法效率极为低下。同时也并非所有的交互矩阵都显式的包含用户的评分。基于模型的协同过滤算法通过对交互矩阵的特征进行学习从而给出推荐结果,如矩阵分解(Matrix Factorization)、隐语义模型(Latent Semantic Models)、及深度学习(Deep Learning)。

矩阵分解是最常用且高效的一种实现方法方法,例如通过将交互矩阵通过SVD分解得到用户和物品的潜在向量特征表示,这些特征相比于原先的交互矩阵相比有更小的维度。

具体而言,假设我们有一个用户-物品评分矩阵 \( R \),其中 \( R_{ui} \) 表示用户 \( u \) 对物品 \( i \) 的评分。矩阵分解方法通过将矩阵 \( R \) 分解为两个低维矩阵 \( P \) 和 \( Q \),即:

\begin{equation}
R \approx PQ^T
\end{equation}
其中, \( P \) 是用户的潜在特征矩阵, \( Q \) 是物品的潜在特征矩阵。对于用户 \( u \) 对物品 \( i \) 的预测评分 \( \hat{R}_{ui} \) 可以表示为:

\begin{equation}
\hat{R}_{ui} = P_u^T Q_i
\end{equation}

为了找到最佳的 \( P \) 和 \( Q \),我们可以通过优化以下损失函数:

\begin{equation}
L = \sum_{(u,i) \in R} (R_{ui} - P_u^T Q_i)^2 + \lambda (\|P_u\|^2 + \|Q_i\|^2)
\end{equation}
其中, \( \lambda \) 是正则化参数,用于防止过拟合。基于矩阵方法由于其简单高效,并且随着高性能计算技术不断发展,这一过程的计算效率也在不断提升。本工作中基于用户行为的推荐优化业主要基于此算法。

综上,基于用户交互的推荐系统更多的关注于用户交互行为,从行为中提取行为特征并给出预测,往往不需要关注推荐内容本身的特征,因此在面临全新的推荐环境时会面临交互数据缺乏稀疏产生的冷启动问题。同时用户偏好短期产生变化时,模型固定可能无法及时产生长短期变化的推荐结果,也是需要着重优化的目标。


\subsection{基于内容的推荐系统}

推荐系统最早的思想就是基于内容本身进行推荐,但碍于特征提取技术发展缓慢,并没有基于用户交互行为的推荐系统高效。Gerard Salton\cite{salton1989information}首次提到了信息检索(Information Retrieval)与信息过滤(Information Filtering)之间的关联与区别。前者主要关注在大型数据库中被动提供检索结果,而后者则是现推荐系统的雏形。信息过滤系统会持续检测系统中的信息流以及数据变化,并且主动的提供检索结果给用户。在Gerard在信息过滤技术里提出了向量空间模型(Vector Space Model),开创性的将内容以及检索都映射到了向量空间中,检索结果由检索向量以及内容向量之间的相似度给出。

因此基于内容的推荐算法的核心就是对用户和物品本身进行特征提取\cite{pazzani2007content},利用向量嵌入技术(embadding)将用户、物品以及检索行为映射到向量空间,以用户向量化为例,通过分析用户档案及交互记录来构建用户偏好的模型,可以表示为一个加权向量:

\begin{equation}
\mathbf{p} = (p_{1}, p_{2}, \ldots, p_{m})
\end{equation}
其中,$p_{j}$ 表示用户对特征 $j$ 的偏好权重。在得到不同用户的向量表示之后就可以计算相似度,相似度的计算是基于向量的余弦相似度,

\begin{equation}
\text{sim}(\mathbf{p}, \mathbf{x}_i) = \frac{\mathbf{p} \cdot \mathbf{x}_i}{\|\mathbf{p}\| \|\mathbf{x}_i\|}
\end{equation}
其中,$\mathbf{p} \cdot \mathbf{x}_i$ 是向量的点积,$\|\mathbf{p}\|$ 和 $\|\mathbf{x}_i\|$ 分别是向量的范数。


向量点积的公式为:
\begin{equation}
\mathbf{p} \cdot \mathbf{x}_i = \sum_{j=1}^m p_{j} \cdot x_{ij}
\end{equation}

向量范数的公式为:
\begin{equation}
\|\mathbf{p}\| = \sqrt{\sum_{j=1}^m p_{j}^2}
\end{equation}
\begin{equation}
\|\mathbf{x}_i\| = \sqrt{\sum_{j=1}^m x_{ij}^2}
\end{equation}

为了生成推荐,系统为每个项目 $i$ 计算一个得分,然后根据得分从高到低排序,选择得分最高的项目推荐给用户。基于内容的推荐系统的核心就是特征提取及向量化,这一部分的介绍及优化将在第二章及第三章着重展开。因此基于内容的推荐系统很大程度上依赖于特征提取以及向量化的质量。尽管如此,基于内容的推荐系统算法有着很多独特的有点。首先推荐结果能够更加个性化,因为结果基于对用户及交互内容的语义分析而非统计特性。其次由于向量化模型本身蕴含了隐藏特征,因此及时面对全新的推荐系统或跨领域推荐,也可以极大程度的缓解冷启动问题。同时,由于基于内容的推荐系统保留了很丰富的语义特征,因此推荐系统整体的可解释性也是比较强的。


\subsection{基于大语言模型及其智能体的推荐系统}
随着大语言模型技术的发展,一系列自然语言处理领域的公开问题得到了解决。主要有基于大语言模型及基于大语言模型智能体的两大类推荐系统。大语言模型例如ChatGPT、Llama等在推荐系统中凭借出色的语义理解以及长序列预测能力使其能够帮助推荐系统更好地获取语义特征并在较长序列上给出较准确的预测效果。大语言模型能够为推荐系统带来性能提升,主要体现在通过大量语料训练得到预训练模型能够提供准确的语义理解以及高质量的特征提取,例如可以从用户的档案、评论交互、交互列表中准确提取用户兴趣;此外大语言模型一定程度上可以兼顾基于行为以及基于内容的推荐系统,从而在各自应用场景中取长补短,具体而言,类似Transformer\cite{vaswani2017attention}的架构能够给出基于长序列的序列化预测,可以很好的基于交互序列给出继续预测的结果;同时对于内容的强大语义理解也可以在缺乏交互数据或交互矩阵稀疏的情况下很大程度上基于语料库知识缓解冷启动问题。基于大语言模型或智能体的推荐系统已经有一系列颇具影响力的成果发布,例如基于Bert模型的Bert4Rec\cite{sun2019bert4rec}序列预测推荐系统、基于Transformer架构的Transformer4Rec\cite{khorasani2021tf4rec}、基于大语言模型智能体的GPT4Rec\cite{li2023gpt4rec}等等。

\begin{figure}
    \centering
    \includegraphics[width=1\linewidth]{figures/image.png}
    \caption{基于大语言模型的推荐系统示意图,该图来自文献\cite{xu2024prompting}}
    \label{fig:enter-label}
\end{figure}

事实上大语言模型在推荐系统中的角色\cite{xu2024prompting}并不是单一的。整体而言,大语言模型在推荐系统中可以用于语义理解,直接为用户或物品进行向量嵌入;也可以利用其语义理解能力总结概括用户意图,在不同的推荐场景之下进行混合意图识别并进行不同的推荐;还可以基于上下文理解对已经召回的物品做重排、粗排及细排等操作;也可以作为代理人(agent)进行流程控制等。

大语言模型能力强大且具有极强的泛化能力,基于大语言模型的智能体能够利用其强大的推理能力,对传统的推荐系统各个环节存在的短板进行优化,也可以综合行为模型和内容模型的有点进行融合推荐。因此基于大模型进行行为-内容混合架构设计是可行的,通过大语言模型或其智能体在不同的环节参与推荐召回并且承担不同的角色,对用户内容以及交互上下文进行更深入的语义理解并给出更高质量的推荐结果,是本工作的重点以及未来工作需要着重优化的目标。

\subsection{混合推荐系统框架}
基于上述的多种推荐系统以外,还有将多种推荐系统进行混合从而综合提升
性能、结合多种推荐系统的优点并一定程度上规避各自缺点的混合推荐系统架
构\cite{ko2022survey}。常见的混合策略\cite{burke2002hybrid,burke2007hybrid,burke2000knowledge,rendle2012factorization}主要包括加权融合、切换混合、级联融合、混合模型、特征组合、特征增强和元级混合等方法。

基于上述混合推荐系统技术,本工作提出了一种利用大语言模型对内容和行为进行融合的混合推荐系统框架及其实现。

具体而言,在基于用户行为的推荐系统中,主要使用了不同注意力范围的线性加权网络从而可以调整兴趣特征提取范围,并结合大语言模型智能体的推理能力进行模型切换混合。

在基于内容的推荐系统中主要使用大语言模型智能体的上下文推理预测能力,对用户进行兴趣理解以及未来短期兴趣预测,产生相关检索词进行特征增强,再利用大语言模型智能体进行模型调度从而给出推荐结果。

在最终的行为-内容融合架构中主要使用了模型间对齐修正的混合技术,在基于内容进行物品召回以后,对结果增加行为特征混合从而进行行为特征的补充,从而更加全面的总结用户交互行为及内容特征并综合给出推荐结果。



 
\section{研究难点}

如上一节综述中所提及的,在推荐系统问题上,已经有了基于用户交互、基于内容以及最新出现的基于大语言模型以及种种算法互相混合的多种策略。然而,将大语言模型很好的融入已有的推荐系统框架任然是一个极为前沿的问题。具体而言,各个推荐系统算法经常会有各自的局限性:基于用户交互行为的推荐系统算法及其依赖大量的先验交互数据以及由此带来的冷启动问题;基于内容的推荐系统算法会受限于特征提取质量以及特征泛化问题;基于大语言模型的推荐系统受限于算力以及远程运算资源开销,难以成为单一的推荐系统组成。这些工作也正说明,急需一种将大语言模型涵盖其中,能够很好的综合已有推荐系统优点,同时能够以合理的开销让大语言模型在关键节点释放性能的平衡架构。

为此,本工作设计了一套基于大语言模型的混合推荐系统框架,并且对框架的不同部分基于大语言模型分别进行优化,这一问题的研究难点主要可归纳如下:

第一,基于用户交互行为的模型缺乏对于推荐对象个性化的推荐修正,用户存在长短期兴趣偏差,而调用模型是统一的,因此将在部分用户的推荐结果上产生较大偏差。而由于基于用户交互行为本身缺乏对内容本身的理解,识别兴趣变化会产生额外的开销,现代推荐系统中的交互矩阵规模是十分庞大的,将会带来巨大的计算开销。

第二,基于内容的推荐系统能够更好的提取用户及物品特征,但是由于不包含交互行为特征,因此对于一些实时热点产生短时间激增的大规模类似交互行为无法产生响应,因而产生推荐偏差。同时推荐过程里若召回数量过大则会导致单次计算开销增大,而最终也需要精细排序,因此会产生较大的无用计算资源开销,减缓推荐系统整体性能。

第三,大语言模型在推荐系统工作流中的定位模糊以及范式缺乏。推荐系统本身是一个庞大的工程系统实现,同时也需要实现一整套具备信息检索功能的系统。在大语言模型加入后,需要尽可能的在语义理解及特征提取的过程里发挥大语言模型的性能,同时也不使系统计算开销及复杂度大幅度提升,同时也需要对齐基于交互行为及基于内容的推荐系统。因此需要一套普适性更强同时平衡性能提升及开销的适用于不同数据集的推荐系统框架及大语言模型基座。


\section{研究方法}
基于第 1.2 部分中已有的相关研究,针对第 1.3 部分提出的研究难点,本工作提出了一套基于大语言模型的混合推荐系统框架,融合基于行为的推荐系统以及基于内容的推荐系统,强化推荐系统的语义理解以及特征提取功能,增强了整体模型的交互预测能力,研究方法主要包含以下内容:

第一,本工作设计了基于用户交互及大语言模型智能体的长短期兴趣变化识别及自适应推荐组件,并在传统的矩阵分解算法上增加了权重调整的线性层使得模型的注意力范围可以人为控制从而适应长短期推荐的不同情景。同时设计了智能体的记忆更新机制,能够更准确的识别出用户兴趣变化情况并且一定程度上细分量化出变化程度并作出模型调度的决策,从而弥补了传统基于行为的推荐系统对于内容理解的缺失以及模型更新的开销过大。

第二,本工作基于Li等人的GPT4Rec\cite{li2023gpt4rec}架构基础上修正了某些由于内容中过长干扰项产生的向量化偏差以及相似度计算误差。本工作在此基础上提出了基于正则阈值过滤的Query-Search框架用于对基于内容的推荐系统的特征识别增强。依靠大语言模型的语义理解及泛化能力,生成未来检索词预测并快速生成粗排结果用于后续细化,规避了传统算法中计算资源浪费的问题。

\begin{figure}[t]
    \centering
    \includegraphics[width=1\linewidth]{figures/GPT4REC.png}
    \caption{GPT4Rec流程示意图,该图来自文献\cite{li2023gpt4rec}}
    \label{fig:enter-label}
\end{figure}

第三,本工作提出了一整套基于大语言模型的完整的混合推荐系统对齐框架,通过锚定基于内容推荐系统推荐结果并利用大语言模型的语义理解能力用基于交互的长短时自适应网络进行修正,从而使得最终结果能够兼顾交互行为及内容理解,同时也能够节省大语言模型的资源开销。这一框架提出了一种将大语言模型融入现有推荐系统体系的方法,并且在未来存在高度可扩展性及泛化性。

