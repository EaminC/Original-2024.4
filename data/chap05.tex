% !TeX root = ../thuthesis-example.tex
\chapter{总结与未来工作}
\section{内容总结}

本工作基于大语言模型技术以及基于内容及基于行为的推荐系统的分析,分别提出了基于交互数据加权及大语言模型系统的长短期自适应推荐系统、基于阈值过滤以及阈值匹配的内容推荐系统、以及基于大语言模型及两种推荐方法修正对齐的混合意图识别推荐系统框架。具体而言,本工作首先提出了推荐智能体的记忆结构用于形成对推荐系统主体的语义理解和认知。其次,提出了基于矩阵分解的加入线性权重层以及基于大语言模型智能体的模型调用框架用于强化基于用户交互行为数据的推荐流程。随后提出了一种基于阈值过滤以及正则匹配增强的基于内容推荐系统。最后基于上述推荐系统及智能体给出了完整的基于大语言模型的内容-交互修正对齐的混合意图识别推荐系统,并给出了完整的系统设计框架图。

同时本工作基于上述推荐系统框架给出了实验验证,证明了混合意图识别的推荐系统框架对于推荐召回率以及推荐质量评价指标的提升,同时增加了推荐系统的整体可解释性以及语义推理能力。作为大语言模型与传统推荐系统算法融合的前沿课题研究,本工作为基于大语言模型的推荐系统设计提供了一套全新的范式,对于推荐系统领域后续深入研究以及社会经济综合发展都会有更加深远的意义。

\section{未来工作}

未来的工作更加关注基于大语言模型基于真实交互情景的交互式推荐以及更多的基于语义理解推理的推荐技术研究。具体而言现在的推荐系统指标都基于收集的既往推荐系统行为数据以及知识库,而实际上推荐行为本身也是会对被推荐主体的行为产生影响的。基于大语言模型技术以及其他的语义理解技术,有望在之后的推荐系统设计中加入更多与被推荐主体交互以及基于真实推荐环境中产生的实时数据集进行推荐系统技术优化研究,并基于全新的交互场景提出召回率、ndcg以外的更加合适的评价指标。

此外,基于大语言模型的推荐系统能够对更加混合式的推荐环境产生理解。诸如第三章开头所提到的城市空间情景下的推荐问题,相比与网络空间的物品有更多的真实时空信息,因此对POI的推荐问题会更加依赖于对于语义以及自然语言、时空信息的理解能力,也是现在推荐系统技术难以解决的前沿难题。未来基于大语言模型以及智能体技术的研究有望对城市空间推荐技术有更加深入且有价值的研究。